Vim�UnDo�q!Sv�'��|��?Oj�cH����X����Sp]_�����SpZ�\section{Acknowledgments}CThis work was supported in part by the National Science Foundation under grant No. PHY-9424278.\begin{thebibliography}{99}?\bibitem{Huf00} P.~R. Huffman et al., Nature 403 (2000) 62.        A\bibitem{OI} Oxford Instruments, Inc.  Concord, MA. Certain trade@names and company products are mentioned in the text in order toEadequately specify the experimental procedure and equipment used.  InDno case does such identification imply recommendation or endorsementby NIST.	\end{thebibliography}\bibliographystyle{revtex}\bibliography{Lifetime}\end{document}5�_�����Sp\�
0\documentclass[preprint,review,12pt]{elsarticle}Q% \documentclass[review,number,sort&compress]{elsarticle}  % submit with this one\usepackage{lineno}F%% Use the options 1p,twocolumn; 3p; 3p,twocolumn; 5p; or 5p,twocolumn%% for a journal layout:-%% \documentclass[final,1p,times]{elsarticle}7%% \documentclass[final,1p,times,twocolumn]{elsarticle}-%% \documentclass[final,3p,times]{elsarticle}7%% \documentclass[final,3p,times,twocolumn]{elsarticle}-%% \documentclass[final,5p,times]{elsarticle}7%% \documentclass[final,5p,times,twocolumn]{elsarticle}\usepackage{graphicx}\usepackage{amssymb})\journal{Nuclear Instruments and Methods}!\newcommand{\fixme}{{\bf FIXME:}}I\newcommand{\BBN}{Big Bang Nucleosynthesis (BBN)\renewcommand{\BBN}{BBN}};\newcommand{\SM}{Standard Model (SM)\renewcommand{\SM}{SM}}B\newcommand{\UCN}{ultracold neutron (UCN)\renewcommand{\UCN}{UCN}}G\newcommand{\EDM}{electric dipole moment (EDM)\renewcommand{\EDM}{EDM}}\\newcommand{\NCSU}{North Carolina State Univeristy (NC State)\renewcommand{\NCSU}{NC State}}c\newcommand{\NIST}{National Institute of Standards and Technology (NIST)\renewcommand{\NIST}{NIST}}W\newcommand{\NCNR}{\NIST\ Center for Neutron Research (NCNR)\renewcommand{\NCNR}{NCNR}}S\newcommand{\LANL}{Los Alamos National Laboratory (LANL)\renewcommand{\LANL}{LANL}}G\newcommand{\HMI}{Hahn-Meitner Institute (HMI)\renewcommand{\HMI}{HMI}}G\newcommand{\ILL}{Institut Laue-Langevin (ILL)\renewcommand{\ILL}{ILL}}J\newcommand{\CKM}{Cabibbo-Kobayashi-Maskawa (CKM)\renewcommand{\CKM}{CKM}}F\newcommand{\TPB}{tetraphenyl butadiene (TPB)\renewcommand{\TPB}{TPB}}D\newcommand{\EUV}{Extreme ultraviolet (EUV)\renewcommand{\EUV}{EUV}}T\newcommand{\AMS}{accelerator-based mass spectroscopy (AMS)\renewcommand{\AMS}{AMS}}J\newcommand{\SNS}{Spallation Neutron Source (SNS)\renewcommand{\SNS}{SNS}}^\newcommand{\KEK}{High Energy Accelerator Research Organization (KEK)\renewcommand{\KEK}{KEK}}H\newcommand{\DAQ}{data acquisition system (DAQ)\renewcommand{\DAQ}{DAQ}}W\newcommand{\NSAC}{Nuclear Science Advisory Committee (NSAC)\renewcommand{\NSAC}{NSAC}}G\newcommand{\dPS}{deuterated polystyrene (dPS)\renewcommand{\dPS}{dPS}}U\newcommand{\dTPB}{deuterated tetraphenyl butadiene (dTPB)\renewcommand{\dTPB}{dTPB}}C\newcommand{\MCP}{multichannel plate (MCP)\renewcommand{\MCP}{MCP}}L\newcommand{\MTN}{marginally trapped neutrons (MTN)\renewcommand{\MTN}{MTN}}I\newcommand{\UVT}{ultraviolet transmitting (UVT)\renewcommand{\UVT}{UVT}}E\newcommand{\PMT}{photomultiplier tube (PMT)\renewcommand{\PMT}{PMT}}+\newcommand{\zeroplus}{$0^+\rightarrow0^+$}\begin{document}\begin{frontmatter}R\title{Design and Construction of an Apparatus for Magnetically Trapping Neutrons}?%% use optional labels to link authors explicitly to addresses:(%% \author[label1,label2]{<author name>}%% \address[label1]{<address>}%% \address[label2]{<address>}\author[NISTG]{H.\,P. Mumm}\author[NISTG]{A.\,K. Thompson}\author[NISTG]{M.\,G. Huber}\author[NISTG]{K.\,J. Coakley}\author[NISTG]{M.\,S. Dewey}\author[NCSU]{P.\,R. Huffman}!\author[NCSU]{K.\,W. Schelhammer}\author[NCSU]{C.\,R. Huffer}\author[NISTB]{K.\,J. Coakley}\author[UMD]{A.\,Yue}"\author[UNC]{C.\,M. O'Shaughnessy}\author[UIUC]{L. Yang}n\address[NISTG]{National Institute of Standards and Technology, 100 Bureau D. MS 8461, Gaithersburg, MD 20899}V\address[NCSU]{North Carolina State University, 2401 Stinson Drive, Raleigh, NC 27695}Y\address[NISTB]{National Institute of Standards and Technology, ????, Boulder, CO XXXXXX}&\address[UMD]{University of Maryland,}-\address[UNC]{University of North Carolina, }<\address[UIUC]{University of Illinois at Urbana-Champaign, }\begin{abstract}u   The design, construction, and operation of an apparatus used to magnetically trap ultracold neutrons is described.   \end{abstract}   \begin{keyword}3%% keywords here, in the form: keyword \sep keyword3%% MSC codes here, in the form: \MSC code \sep code5%% or \MSC[2008] code \sep code (2000 is the default)
\end{keyword}\end{frontmatter}*% \linenumbers  % will need for submission%\maketitle\tableofcontents\newpage\section{Introduction}KThe decay of the free neutron is the simplest nuclear beta decay and is theMprototype for all charged current semi-leptonic weak interactions.  The decayKparameters, the neutron lifetime in particular, provide essential inputs toHinvestigations of the weak interaction.  A precise value for the neutronGlifetime is required for several internal consistency tests of the \SM\Nincluding searches for right-handed currents and tests of the unitarity of theFCKM mixing matrix.  Measurements of neutron decay coefficients provideNinformation on the vector and axial-vector coupling constants $g_v$ and $g_a$.JThe neutron lifetime is also an essential parameter in the theory of \BBN.BThere is a significant discrepancy between recent neutron lifetimeDexperiments that is not understood.  It is essential to resolve thisIdisagreement.  This can best be accomplished through measurements using aGvariety of techniques having very different systematic uncertainties.  FIn our novel approach, a population of \UCN\ are produced by inelasticMscattering of cold (0.89~nm) neutrons in a reservoir of ultra-pure superfluidN$^4$He (the ``superthermal'' process)~\cite{GOL79, GOL77, KIL87, GOL83}. TheseOneutrons are then confined by a three dimensional magnetic trap. As the trappedOneutrons beta decay, the energetic electron products generate scintillations inLthe liquid He~\cite{STO70, KAN63}, that are detectable with high efficiency.KThe neutron lifetime, $\tau_n$, can be directly determined by measuring the7scintillation rate as a function of time~\cite{GOL94}. NThis paper presents a detailed description of the experimental apparatus built)to perform such a measurement at \NIST\ ."\section{Design and General Setup}IThe apparatus consists of a superconducting magnetic trap to confine freeJneutrons and a light detection system to monitor their decay.  The trap, aI3--dimensional magnetic well, is loaded with \UCN s through the inelasticKscattering of 0.89~nm neutrons with superfluid helium.  A cold neutron beamNfrom the NIST Center for Neutron Research (NCNR) is incident upon the trappingLregion. The \UCN\ scattering products accumulate in the trap potential. WhenNthe trap is saturated, i.e. the rate of decay balances the rate of production,OThe neutron beam is blocked from entering the trap region and the population ofMtrapped neutrons are allowed to decay.  Recoil electrons from beta decay willKionize the helium, producing scintillation light.  The UV light produced isOwavelength shifted to visible and each decay event is recorded as a function oftime.+\subsection{Superthermal Production of UCN}OTraditional \UCN\ sources extracted the small relative population of \UCN\ fromMthe tail of the energy distribution of neutrons in thermal equilibrium with aLmoderator.  This resulted in a relatively high background rate and low \UCN\Jdensity.  Modern \UCN\ sources take advantage of scattering processes thatLbreak the thermal equilibrium to obtain a higher \UCN\ density. This type of>production method is referred to as superthermal production.  NEnergy dissipation occurs in liquid helium (L-He) by quanta of vibrational andMrotational excitations (phonons and rotons). The dispertion relation relatingMthe energy and momentum of these excitations deviates from the kinetic energyJof free neutrons and the two cross at two points only.  This results in anOeffective two state system where a 0.89 nm cold neutron creates a phonon in theNL-He and imparts almost all of its energy.  The reverse process can be limitedNby decreasing the temperature of the L-He bath thereby limiting the populationOof phonons in the L-He.  If the temperature is sufficiently low the probabilityNof upscattering a neutron becomes effectively zero, allowing for \UCN\ storage-times that approach the beta decay lifetime. !\subsection{Detection Principle} \label{sec:DetPrinc}IThe neutron decay detection is based on beta induced scintillation of theMsuperfluid helium used to downscatter neutrons within the trap. The energeticKelectrons ionize liquid helium generating a cloud of ion-electron pairs andLexcited helium atoms. These interact with one another leaving behind excitedOdiatomic molecules of helium in both singlet and triplet states. A continuum ofJextreme ultraviolet (EUV) light is emitted when the molecules relax to theMmonoatomic ground state and the signal from the singlet state is prompt, lessHthan 10~ns. With a wavelength in the range of 60--100~nm (12--20 eV) theMemission is too low of an energy to interact with the helium that has a firstNexcited state around 20~eV. This ensures that the helium is transparent to thescintillation light. OWhile the helium is transparent to the scintillation light, in order to provideMa useful signal, one must transport it out of the measurement region. This isOaccomplished by coating the neutron production and measurement cell region withL\TPB\, an organic flour that downshifts the EUV signal to a pulse of visibleMphotons in the blue wavelengths with an efficiency of 135\%.  This signal canEthen be transported out of the system using standard light collectionItechniques and detected in photomultipliers outside of the apparatus. ForPdetails on the light collection system refer to Sec.~\ref{sec:LightCollection}. 7\subsection{Sensitivity? {\bf Andrew} (Paul will help)}\section{Experimental Setup}=Some general description of the facility, footprint, and etc.\subsection{Cryogenic System}HA measurement cell is filled with superfluid $^4$He for the superthermalKproduction of \UCN. The magnetic trap is made from superconducting wire andNtherefore must be cooled with liquid helium.  The system must accommodate bothKa vertically-oriented dilution refrigerator to cool the measurement cell asGwell as a liquid helium vessel for cooling the approximately 1.5~m longKhorizontally-oriented superconducting magnet.  In addition, the cryostat isOdesigned to accommodate the 0.89~nm neutron beam entering horizontally from oneLend of the magnet as well as a light collection system that transports lightfrom the opposite end.    \begin{figure}[tbph]	\begin{center}6	    \includegraphics[width=\textwidth]{figures/dewar}
	\end{center}g	\caption{Cross-sectional view of the apparatus.  The individual components are described in the text.}	\label{fig:dewar}    \end{figure}IThese constraints led to a design of a horizontal dewar with two verticalOtowers extending upwards at each end that accommodate the dilution refrigeratorNand magnet current leads.  A cross-sectional view of the apparatus is shown inLFigure~\ref{fig:dewar} for reference.  The dewar primarily consists of threeLconcentric shells, each at successively lower temperatures.  The outer, roomJtemperature shell provides the primary vacuum enclosure for the apparatus.LInside of this resides the liquid nitrogen jackets and thermal shields, withMthe liquid helium jacket and shields inside, finally surrounding the dilutionJrefrigerator and measurement cell.  The nominal operating temperatures forNthese shells are 300~K, 77~K and 4.2~K respectively.  The vacuum space betweenMthe 300~K and 4.2~K shields is common and denoted as the outer vacuum chamber(OVC).    JThe apparatus is designed around the horizontally-oriented superconductingImagnet that is housed in the central region of the dewar.  This magnet isLsurrounded by a liquid helium bath and operates at a temperature of 4.2~K. AOthin-walled stainless steel tube extends through the central bore of the magnetNand is used to create a vacuum space throughout the central region.  This tubeKis connected to two larger vacuum spaces at each end of the magnet and alsoLextends upwards to surround a $^{3}$He-$^{4}$He dilution refrigerator.  This;vacuum volume is denoted as the inner vacuum chamber (IVC).   IThe dilution refrigerator and associated gas handling system provides theGprimary cooling for the approximately 20~liters of liquid helium in theOmeasurement cell.  The cell is connected to the dilution refrigerator through aM1.9~cm diameter superfluid heat link.  The thermal conductivity of superfluidKhelium is experimentally determined to be $20 d T^3$ (W K$^{-4}$~cm$^{-2}$)Jbelow $T= 0.7$~K, where $d$ is the diameter of the tube in cm\cite{Pob96}.LUsing this expression and the measured cooling power of the dilution fridge,J1.1~mW of cooling at 300~mK, we calculate that the temperature of the bulkOliquid helium inside the cell should have an average temperature of 200~mK. TheMactual measured temperatures during data collection were closer to 250~mK. AsGdiscussed in Sec. \ref{sec:HeatLoad}, this is likely a result of largerIradiative heat loads through either the neutron entrance windows or lightKcollection system.  Because of the large diameter of the cell, there shouldLvirtually be no temperature gradient across the cell.  During a magnet ramp,Kabout 0.4~J of energy is dumped into the cell, which raises the temperatureNbriefly to about 400~mK, but cools back to base temperature in several hundredseconds.    *\subsubsection{Magnet support and cooling}:The magnet trap has been discussed in detail in a separateLpublication\cite{Yan??}.  In brief, it is an Ioffe-type magnetic trap designLconsisting of a quadrupole assembly that provides radial confinement and twoOsolenoid assemblies with the same current sense that provies axial confinement.MSuch a configuration eliminates zero field regions inside the trap, thereforeKsuppressing the spin-flips of trapped particles\cite{XXX}.  Due to the highJcost of custom-built high current magnets, we used a combination of a highJcurrent quadrupole (on loan from the KEK lab in Japan) and two low currentLsolenoids for the trap.  In inital testing, the quadrupole reached 78~\% andLthe solenoids reached 70~\% of the loadline respectively, corresponding to aMtrap depth of 2.8~T.  We typically ran at 70~\% during these runs to minimizedowntime from quenches.KTo support the magnetic trap while minimizing the heatloads into the liquidJhelium bath, we designed and tested G-10 fiberglass based cryogenic posts.OAmong commonly used cryogenic materials, G-10 has the highest yield strength toHthermal conductivity ratio, about ten times higher than stainless steel.MOur cryogenic post design closely follows the design by Nicol \textit{et al.}Bat Fermilab\cite{Nic87}.  Schematic views of the post are shown inOFigure~\ref{fig:supportpost}.  The main body of the post is a 19~cm long, 19~cmMdiameter G-10 tube with 1~mm wall thickness.  Equally spaced aluminum flangesKpositioned at the bottom (300~K), middle (77~K) and top (4.2~K) of the tubeDprovide lateral mechanical support for the tube as well as holes forMattachments to the dewar.  All flanges were shrunk-fit onto the tube.  The IDJof each outer aluminum ring was machined to the exact OD of the G-10 tube,Mwhile the OD of each inner metal disk was machined to be 0.5 mm more than theLtube ID. During assembly, the outer rings were fitted onto each joint first,Mthen the inner disks were cooled to liquid nitrogen temperature (77~K) beforeLthey were fitted onto each joint.  Small lips on the 300~K and 4.2~K flangesLhelp to register them during assembly.  Side surfaces of the rings and disksNwere roughened slightly using sand paper to increase the friction coefficient.OThe 77~K and 4.2~K disks were covered with super-insulation to reduce blackbodyKradiation.  The internal volume of the post was vented through holes in theLmetal disks and G-10 screws, because tests showed that any vent holes on theMside of the G-10 tube significantly reduced the load carrying capacity of thepost.    \begin{figure}[t]	\begin{center}*	    \includegraphics{figures/supportpost}
	\end{center}f	\caption{Two dimensional and three dimensional cross-section views of the G-10 based cryogenic post.}	\label{fig:supportpost}    \end{figure}OA test post was built based the above design.  It was load tested up to 1360~kgHat room temperature.  The heatload through the post, though not measuredNdirectly, was calculated from available G-10 thermal conductivity data and was-estimated to be $(0.35 \pm 0.05$)~W per post.NThe magnet is supported from two posts as seen in Figure~\ref{fig:dewar}.  DueHthe thermal contraction, one post is allowed to move to minimize lateralKstress.  A thin sheet of Teflon is used at the 300~K connection.  ThermallyKlinked to the bottom of the helium volume is a long OFHC-grade copper plateJ25.4~cm (10.0'') wide and 9.53~mm (3/8'') thick that spans between the twoLcryogenic support posts.  This copper bar thermally links the top of the twoGsupport posts and allows one to extract the heat without depositing theOmajority of it into the liquid helium volume.  To remove the heat load from theOsupport posts, a Gifford McMahon-type cryocooler that provides 1.5~W of coolingJpower at 4.2~K is thermally linked to middle of the copper bar.  OFE-gradeJcopper braiding is used to thermally link the cryocooler to the copper barNwhile allowing freedom for thermal contraction as the apparatus cools.  CopperNbraids also connect this bar to the neutron entrance and light exit windows to&minimize heatloads from these as well.    %\subsubsection{Vertical Dewar Towers}LAs the dilution refrigerator uses gravity to set up the phase boundaries andOthermal gradients, it must operate in a vertical orientation.  Our refrigeratorMis housed in the left-most tower as is shown in Figure~\ref{fig:dewar}.  ThisOportion of the dewar is a standard nitrogen-jacket cooled vertical dewar with aLseparate liquid helium bath.  The bottom is specially designed to connect toCthe horizontal section while providing a common IVC that houses theKliquid-helium heat link that extends to the cell.  The helium bath providesNcooling for the refrigerator as well as helium for operation of its pumped 1~KMpot and is separate from the helium that surrounds the magnet.  {\bf CHECK --OThe liquid helium boiloff associated with this tower is approximately 30~l/day.&-- Here or in Sec. \ref{sec:HeatLoad}}    MThe right-most vertical tower in Fig. \ref{fig:dewar} provides access for theIhigh-current magnet leads extending to the magnet as well as serving as aIhelium reservoir for the magnet itself.  Introducing 3400~A into a liquidMhelium bath required the use of high temperature superconducting (HTS) leads.KA prototype set of leads developed at Fermilab and graciously loaned to theLexperiment are described in detail in reference \cite{Cit99}.  The leads, asKshown in Figure~\ref{fig:HTSleads}, are constructed in three sections.  TheMupper section is copper and provides a thermal gradient from room temperatureEto the HTS section.  The middle section consists of parallel tapes ofEmultifilamentary superconductor in a silver alloy matrix.  Nb$_{3}$SnNlow-temperature superconductor (LTS) leads are then connected to the lower HTSKsection through a copper section that also acts as a thermal link to 4.2~K.        \begin{figure}[t]	\begin{center}) 	    \includegraphics{figures/HTS-Leads}
	\end{center}Z	\caption{Cross-sectional views of the high temperature superconduting leads\cite{Cit99}.}	\label{fig:HTSleads}    \end{figure}    MSince the HTS material must be kept below 80~K to remain superconducting, theKcopper -- HTS junction is cooled with a continuous flow of liquid nitrogen.MCold nitrogen gas is flowed at a rate of 0.7~g/s (70~scfh) through the leads,Nconsuming about 70~l/day.  To hold the temperature below the critical point ofMthe HTS material one must also maintain a flow rate of 0.026~g/s (19 scfh) ofLcold helium boil-off vapor through the current leads.  This corresponds to aNhelium consumption rate of 16~l/day.  To minimize heat loads as well as reduceOthe dependency on vapor cooling, a second cryocooler is thermally linked to theLbottom copper section of the current leads to provide an additional 1.5~W ofOcooling power at 4.2~K. A thin layer of Kapton is used to provide the necessaryelectrical isolation.    LIn the event that the magnet quenches, a large fraction of the liquid heliumLwill boil as a portion of the magnets' energy is dissipated in the cryogenicIbath.  A large safety vent is installed on the magnet lead tower for thisexhaust.(\section{Neutron Beam and Monochromator}\label{sec:NeutronBeam}KThe neutron lifetime experiment is performed at the NIST Center for NeutronHResearch (NCNR).  The NCNR operates a split-core 20 MW research reactor.KNeutrons are released through fission reactions of $^{235}$U in the reactorOcore . The fission neutrons (1 MeV) are thermalized to room temperature (300 K,M26 meV) by a heavy water (D$_2$O) moderator surrounding the core. In additionLto thermal neutrons,  the facility has a 20 K liquid-hydrogen ``cold source"Kthat produces cold neutrons with a Maxwellian energy spectrum equivalent to34~K. EEight neutron guides (NG) transport the cold neutrons to experimental>end-stations tens of meters away.  The cold neutron guides areCrectangularly shaped, 15~cm tall and 6~cm wide.  The left and rightBsurfaces of the guide are coated with $^{58}$Ni, while the top andCbottom surfaces are coated with m=2 supermirror. Cold neutrons withDperpendicular energy less than the Fermi potential of $^{58}$Ni (335?neV) will be totally reflected by the wall, and travel down the?guides at the same glazing angle with minimal loss. The neutron1guides are evacuated to eliminate air scattering.MThree beamlines are available for fundamental physics experiments at the exitGof the NG-6 neutron guide,  one polychromatic (white) beam line and twoNmonochromatic beamlines.  The neutron trapping apparatus resides on the NG-6U,Ma 0.89 nm monochromatic beamline.  Because the superthermal production of UCNKhas a resonance at this neutron wavelength, the signal to background of theFexperiment can be significantly improved if only neutrons near the UCNGproduction peak are introduced into the apparatus.  Producing such longOwavelength monochromatic neutrons requires a monochromator with lattice spacingMof at least 0.45 nm, considerably larger than the 0.335 nm lattice spacing ofKcommonly used graphite crystals.   To achieve the large lattice spacing, weKapplied potassium intercalation techniques and developed a  special neutronKmonochromator  with lattice spacing of  0.874 nm.  This monochromator has aMtotal size of 6 cm $\times$ 15 cm.  It  is constructed from nine tiled piecesNof stage 2 potassium- intercalated graphite with mosaics between 1.1$^{\circ}$Land 2.1$^{\circ}$ and reflectivities of (73 - 91)\% at 0.89 nm.  ImmediatelyMupstream of the NG-6U monochromator is a second graphite crystal set an angleNsuch to filter out the $\lambda/2$ component (0.45 nm) of the NG-6U beam.  TheOneutron flux measured in the center of the monochromatic beam 130 cm downstreamNfrom the monochromator is found to be 3.0 $\times$ 10$^6$ n cm$^{-2}$ s$^{-1}$Kafter its installation in 2001.   The neutron flux of the beamline has beenMchecked several times over the years and no deterioration of the neutron fluxOhas been observed.    Overall, the monochromator reflects more than 80\% of theJincident 0.89 nm neutrons,while reflecting less than 2\% of the total coldLneutron beam. The signal to neutron-induced-background ratio in the magneticKtrapping experiment should be improved by a factor of 40 through use of theMmonochromator. More detailed discussions of the development, characterizationJand installation of the monochromator can be found in Refs.~\cite{Mat04}. \subsection{Neutron Entrance}\label{sec:neutronentrance}LThe beam entrance performs multiple functions. Mainly, it allows the neutronGbeam to pass into the cryostat with minimal attenuation. Meanwhile, theHmaterials used must not be easily activated by the neutron beam. NeutronOshielding materials that produce only prompt gammas are used to screen the restJof the apparatus from the neutron beam.  The window must also be opaque toJblackbody radiation from room temperature surfaces to limit heating of thecell. 4A schematic of the neutron beam entrance is shown inKFigure~\ref{fig:BeamEntrance}. There are three vacuum windows that are eachGmade from 20~mil ($508~\mu$m) thick perfluoroalkoxy (PFA) flouropolymerLfilm\cite{But98}. PFA is typically manufactured very purely without metallicJcontaminants that could activate and produce a source of background in theOapparatus.  Since PFA is a soft material that remains pliable even at cryogenicGtemperatures it can be used as a gasket material to create self sealingOwindows(REF) for the beam entrance on the room temperature 300~K vacuum flange,Gat 4~K between the OVC and IVC, and on the end of the experimental cellKcontaining superfluid helium. These windows efficiently pass neutrons, have6minimal activation, and provide reliable vacuum seals.IPFA is not an ideal material to block blackbody radiation. Two additionalKwindows made from beryllium\footnote{The beryllium foils are purchased formKBrush Wellman Engineered Materials, Freemont, CA.} foil are included in theNbeam entrance. One at 77~K blocks blackbody radiation from 300~K to 4~K, and aHsecond at 4~K blocks the radiation originating at 77~K from reaching the?experimental cell.  Beryllium activates with a long lifetime ofO$6.83\times10^{13}$~s, for this reason the background from beryllium activationOis essentially a constant background, and introduces no systematic shift in theNneutron lifetime. For a discussion of the activation analysis of the berylliumHfoils and the identification of trace impurities see Ref.~\cite{Mat02}. \begin{figure}[!b]\begin{center};\includegraphics[width=0.7\textwidth]{figures/BeamEntrance}\end{center}M\caption[A side view of the neutron entrance and shielding materials.]{A sideOview of the neutron entrance and shielding materials. The boron nitride is used>to shield the apparatus from neutrons to minimize activation.}\label{fig:BeamEntrance}\end{figure}EAs the neutrons enter the apparatus, the entire neutron beam having aLdivergence of $~\pm 1^\circ$ illuminates the entrance windows. Scattered andMhighly divergent neutrons are captured in the surrounding materials and couldIproduce backgrounds to the experiment.  Materials exposed to the beam areHchosen such that the only by-product of the capture reactions are promptIgammas.  The predominant capture process used to absorb neutrons that areDscattered out of the beam is $\rm ^{10}{\rm B}+n\rightarrow ^{7}{\rmILi}+\alpha$. In this reaction the $^{10}$B captures a neutron becoming anIexcited state of $^{11}$B for a short time $\sim10^{-12}$~s upon which itJfissions producing $^7$Li, an $\alpha$ particle, and a $\gamma$ in 94\% ofMreactions. Although the reaction produces a large signal in the apparatus, itOis entirely prompt and therefore not a background during the observation periodHapproximately 300~s after the beam is blocked. Natural boron contains anNabundance of $^{10}$B of $0.199$. There is a sufficient amount of $\rm ^{10}B$Kin natural Boron Nitride\footnote{Saint Gobain Ceramics Grade AX05} (BN) toJcompletely shield the apparatus from the neutron beam.  In the flight pathKbetween 300~K and the cell, one must construct a shield for the surroundingMapparatus that is hermetically sealed to neutrons, while breaking the thermalLgradient between layers of the cryostat with vacuum gaps. To accomplish thisKflight tubes are interleaved as shown in Figure~\ref{fig:BeamEntrance}. TheGvacuum space between layers is minimized so that a neutron must reflectImultiple times before it could escape the flight tube. The probability of7neutron capture on boron in this region is near unity. \section{Magnetic Trap}\label{se:MagneticTrap}MThe superconducting Ioffe-type magnetic trap is an essential component of theJlifetime apparatus,  because the number of trapped neutrons scales roughlyMlinearly with trap volume and with (trap depth)$^{3/2}$.   Over the years, weEhave built three generations of Ioffe traps (Mark I, II and III) withKincreasing trap depth and trap volume.  Details of the design and quenchingMbehavior of these traps can be found in Ref.~\cite{Yang08}.   Here we brieflyMdescribe the main characteristics and performance of the Mark III trap, whichFis the largest and the deepest trap, and is currently installed in theexperimental apparatus. OThe Mark III trap consists of a high-current accelerator type quadrupole magnetFon loan from KEK laboratory in Japan ~\cite{Tsu91} and two low currentMsolenoids. It has a designed trap depth of 3.1 T, a trap volume of 8 l and isOcapable of trapping up to 60,000 UCNs per loading cycle.    The main parameters:of the trap are listed in Table~\ref{tbl:Trap_parameters}.\begin{table}[!h].\caption[]{Main KEK magnetic trap parameters.}&\label{tbl:Trap_parameters} \centering\begin{tabular}{ll}
    \hline/    \bf Parameter                           &\\
    \hlineB    KEK Quadrupole operating current, A   & 3405 (90\% loadline)\\+    KEK Quadrupole inductance, mH   & 58 \\9    Solenoid operating current, A & 225 (75\% loadline)\\!    Solenoid inductance, H & 7 \\    Trap Length, cm & 75.0 \\    Trap radius, cm & 5.5 \\)    Trap depth, T                  &3.1\\
\end{tabular}\end{table}%NAt operating current, the energy stored in the KEK quadruple magnet is 336 kJ,Jwhile the energy stored in each solenoid is 180 kJ.  To protect the systemOagainst catastrophic failure due to magnet quench, it is necessary to implementMquench protection mechanisms.   For the KEK magnet, we built an active quenchJprotection circuit which can quickly detect and dump more than 90\% of theOstored energy in an external dump resistor during a quench.  Due to their largeLinductances, it is not possible to dump the energies stored in the solenoidsNeffectively into external dump resistors.   Thus, they are protected passivelyMby diodes across six subdivisions of each solenoid.  Because the wires in theOsolenoids experience additional Lorentz force due to fields from the quadrupoleImagnet, we designed special magnet forms for the solenoid support and setGrunning current at 75\% of the load line to reduce quench probability. OThe Mark II trap was first tested in a vertical test dewar in 2005.  It reachedHabout 90\% of the design current after two quenches.  After the trap wasNinstalled in the apparatus, It again reached about 90?\% of design current andIran stably for several months before a quench damaged something ( currentJleads? overpressure?)   Afterwards, the trap is run at 70?\% of the design(currents to increase the safety margin. +\subsubsection{Magnetic Field Compensation}DThe experimental hall where the apparatus is located is also home toLexperiments that may be sensitive to magnetic fields. Notably, the spin-echoLspectrometer uses precession of polarized neutrons about a magnetic field toLachieve high resolution measurements of the energy transfer during a neutronUscattering process\cite{?}. A magnetic field compensation coil was designed to reduceEthe stray magnetic fields from the magnetic trap and is optimized forMcancellation at this instrument, located a distance of 15-17~m upstream alongMthe neutron guide. The magnitude of the solenoid falls off as $\propto1/r^3$,Nas opposed to the quadrupole field that falls off as $\propto1/r^4$. Thus, oneEneeds only to compensate for the solenoid fields at large distances. LThe optimal current and number of turns for a solenoidal compensation magnetAlocated on either end of the trap and sized according to existingOinfrastructure was calculated based on the fringe fields from the magnetic trapG(see Section~\ref{sec:neutronsims}).  It was found that compensation isHoptimized when two solenoidal magnets are energized, each with a currentOopposing the trap solenoids of 21500~A-turns.  In the model, this configurationNreduces the fields at the spin-echo instrument from an uncompensated 100~mG to
$<$3~mG.  OThe room temperature copper coil ares powered by a model EMHP 40-600-4111 powerJsupply. This supply can provide a maximum current of 600~A at a voltage ofG40~V. To minimize resistive heating copper bars with a cross section ofO0.635~cm by 3.175~cm were used to create the current loops. The number of turnsIrequired for each coil was 39, at a current of 540~A. The copper bars areOstacked in rectangular layers in the style of a log cabin with the current pathHdefined using insulating kapton film or conductive grease to insulate orNconnect adjacent bars. The geometry is realized by constructing two concentricNcoils on each end of the cryostat. The outer coil has 20 turns and encloses anMarea of 4.05~m$^2$ and is supported from the floor. A second coil of 19 turnsKand an area of 3.70~m$^2$ is placed inside the first and is supported using-12.7~mm (0.5'') thick acetal plastic sheets. LThe coils are held together using stainless steel threaded rods electricallyMinsulated by nylon sheaths having a heat deflection temperature at 1.8~MPa ofN$60~^\circ{\rm C}$\cite{matweb}. An interlock system was designed to eliminateKthe danger of the magnet shorting above this temperature. Thermocouples areNinstalled on the upper corner of each magnet.  The voltages are monitored withLtwo thresholds. The first threshold produces a TTL logic output to allow theIDAQ to notify someone that there is a problem with the system. The secondKthreshold will inhibit the compensation coil power supply, and output a TTLKlogic signal to the DAQ that initiates the magnetic trap  power supplies to4ramp the trap current down in a controlled manner.  +\subsection{Gas handling system {\bf Mike}}\subsection{Light Detection}\begin{figure}[!t]\begin{center};\includegraphics[width=\textwidth]{figures/LightCollection}\end{center}T\caption[A top view of the new light collection system.]{A top view of the new lightOcollection system. Illustrated here are the cryostat in black, the experimental6cell in red, and the light collection system in blue.}\label{fig:LightCollection}\end{figure}KThe light collection system of the neutron lifetime apparatus consists of aInumber of optical elements, shown in Fig.~\ref{fig:LightCollection}, thatOtransport photons from the experimental cell to \PMT s where they are detected.OThe previous apparatus was first modeled to benchmark the code using previouslyKcollected data. Each of the components are then assessed independently suchJthat the largest losses may be reduced in the design of the upgrade to theapparatus. KDecay electrons interact with the superfluid helium\cite{McK02} and produceK\EUV\ photons, as described in Sec.~\ref{sec:DetPrinc}, in a quantity of 22Ophotons per keV of beta energy.  The helium is transparent to its scintillationNlight, thus these photons propagate isotropically to the inner surfaces of theOcylindrical experimental cell. The cell is lined with {\sc G}ore-{\sc T}ex withLa thin coating of \TPB\ evaporated on its inner surface. This element servesKtwo functions. First, the TPB is an energy downconverter\cite{McK97,McK04},Labsorbing the EUV photons and emitting visible light photons with an averageOwavelength of 440~nm, having a fluorescence conversion efficiency of 1.4.  ThisJprocess converts the \EUV\ decay signal to one that may be collected usingOconventional optics. Secondly, the detector insert acts as a diffuse reflector,Ltransmitting the light to either end of the cell through a series of diffuse
reflections. MPhotons reaching the downstream end of the cell pass through an \UVT\ acrylicLlight guide that transports them to the cell end. The index of refraction ofHliquid helium at the operating temperature\cite{Edw56} is roughly 1.029,Otherefore total internal reflection in the acrylic is the same as if it were inIair. Upon leaving the light guide the photons pass through a small gap ofLliquid helium, through a 0.635~cm thick acrylic optical window, and exit theLmeasurement cell. The previous apparatus included a $\rm B_2O_3$ Boron OxideMbeam stop before the \UVT\ lightguide, See Fig.~\ref{fig:OldCollection}, as aOprecaution against color center formation that could impede the transmission ofNvisible light. It was subsequently found to be unnecessary and so omitted fromthis design.OThe remainder of the light collection system consists of an acrylic optical andMvacuum window at 4~K, and an additional quartz window in the same location toLremove the heat load to this surface from blackbody radiation originating atMthe 77~K surfaces. An acrylic light guide extends from 77~K to 300~K.  In theNMark II apparatus this element was limited in length and therefore required anFadditional surface of sapphire at the 77~K end to remove the heat load!conducted from room temperature. NA monte-carlo model based on an optical photon ray tracing code called GuideItLwas used for optimization of the light collection system.  This program usedOthe geometry and a number of material parameters to propagate photons through aKseries of optical elements.  Input parameters for these simulations are theFvalues of the index of refraction, the surface roughness, and the bulkNattenuation length for each of the materials. Indices of refraction are neededNfor calculating both the transmission and reflection probabilities, as well asMthe refraction angle. The bulk attenuation length is the length for which theEintensity is reduced by a factor of $1/e$.  The surface roughness andIspecularity parameters apply additional loss at surfaces due to imperfectreflections.LThe collection efficiency is calculated as the number of photons passing theGlast surface into the \PMT s normalized by the number of source photonsMcreated. The measured detection efficiency on the other hand is the number ofOphotoelectrons (p.e.) counted in a single \PMT\ on the end of the apparatus forOeach of the 364~keV events created by a $^{113}$Sn beta source.  Details of theAelements of the light collection system modeled here are shown inDFigure~\ref{fig:OldCollection} and can be found in Ref.~\cite{Yan06,DzhThesis}. \begin{figure}[!t]\begin{center}L\includegraphics[width=0.7\textwidth, angle=180]{figures/DzhosyukLightguide}\end{center}O\caption[Light Collection System from the previous apparatus.]{Light CollectionMSystem from the previous apparatus\cite{DzhThesis}. Calibration data taken inIthis apparatus was used as a benchmark for the GuideIt simulation of this
geometry.}\label{fig:OldCollection}\end{figure}HEach beta event is expected to create on average 11,200 visible photons.NAdditionally, at the peak emission wavelength of the \TPB, 430~nm, the quantumOefficiency for the \PMT\cite{burle} is $17~\%$.  After applying the appropriateDscaling to the simulation the results are in good agreement with the:calibration data taken in the previous apparatus, shown inFigure~\ref{fig:efficiency}.MHaving completed the benchmark with prior data, a new light collection systemRwas designed without the saphhire and Boron Oxide windows. As described in SectionO\ref{sec:magnets}, the new trap is both larger radius and longer. Simulation ofKthe new design are shown in red in Figure~\ref{fig:efficiency}. The averageJgain in photoelectrons per event will be significantly higher. The averageLphotoelectron peak position of the new collection system is a factor of 1.64Bgreater than the average peak position of the previous apparatus. \begin{figure}[!t]\begin{center}A\includegraphics[width=0.8\textwidth]{figures/DetectorEfficiency}\end{center}M\caption[Detection efficiency as a function of axial position within the cellKof the new apparatus.]{Detection efficiency as a function of axial positionKwithin the cell.  Data from events generated using a $^{113}$Sn line sourceI(purple) confirms the validity of the GuideIt model (green) on the MarkIIDapparatus geometry, the upgraded geometry of the KEK Trap predicts aNsignificant improvement both with (red) and without (blue) a reflective windowas described in the text.}\label{fig:efficiency}\end{figure}\subsection{Shielding}D\subsubsection{External shielding and active muon veto {\bf Pieter}}"\subsubsection{Magnetic shielding}!% I intend to cut this down - KWSOThe \PMT\ gains are sensitive to magnetic fields as the electron multiplicationOpath through the series of dynodes can become defocused.  A multi-layer passiveKshielding solution was implemented as shown in Figure \ref{fig:PMTHousing}.NThree layers of passive $\mu$-metal, with high relative permiability, surroundNand thus help to divert the magnetic fields around the \PMT s. A 1.6~mm thick,O432~mm long sleeve slides over each \PMT\ directly outside of the dynodes.  TheK\PMT s are each enclosed in a nitrogen purged aluminum housing.  Outside ofNthis enclosure two 3.2~mm thick, 546~mm long $\mu$-metal sleeves surround eachO\PMT\ in their entirety.  These were fabricated with one end closed to minimizeOfield leakage from the rear of the enclosure.  For the outermost layer, a threeKpart box, 6.35~mm thick, encapsulates both \PMT s. A rectangular panel withLseveral holes to pass the lightguide and detection system housing formed theKfront part of the box closest to the measurement cell.  A large rectangularOsleeve with two open ends slides into place over the detection system.  The endLcap of the box is another panel that has two square holes to pass signal andpower cables for the detectors.MIn addition to the passive shielding, two solenoidal coils are used as activeIcompensation to further reduce the magnitude of the magnetic field in theMvicinity of the \PMT s.  A coil near the opening of the box on the front sideMis used to compensate the dipole field at the opening.  Simulations show thatOthis reduces the magnetic field by at least an order of magnitude.  This magnetOis a 300 turn, 10 inch diameter coil that produces a dipole field of roughly 30KGauss at the center of the coil.  The presence of a large iron power supplyOrack for the trap magnets draws field lines closer to one of the \PMT s causingIits gain to be supressed as compared to the other. To compensate for thisNeffect another coil was wrapped around the second layer of $\mu$-metal on thatJside of the detection system. This coil is used to fine tune the fields toLbalance the gains of the two detectors. The tuning was performed for each ofNthe running configurations of the magnets and is set to the optimal current bythe DAQ.1\subsection{System controls and Data acquisition}LThe data acquisition system for this experiment is managed by two computers.GOne machine is used for slow control of the experiment: magnet control,Gexperiment monitoring, and logging of experimental systems.  The secondMcomputer is used exclusively for waveform acquisition and is coordinated with(the slow control through logic signals. \subsubsection{Control}NThe slow control machine uses a National Instruments PCI 6025E general-purposeNdata acquisition card for analog I/O and digital I/O.  Analog I/O applicationsJinclude: control and monitoring of the cancellation coil current supplies,Ncontrol and monitoring of the silicon-controlled rectifiers (SCRs) that bypassNthe quadrupole magnet quench protection resistor during normal trap operation,Cand \PMT\ high-voltage controls.  Digital I/O is used primarily forOcoordination with the \DAQ. The GPIB-interface on the current supplies for bothIthe solenoid and quadrupole magnets of the trap are accessed via PrologixMGPIB-USB controllers from the slow control machine.  A high-voltage isolatingGUSB hub is needed between the GPIB-USB controllers and the slow controlMcomputer because the magnet power supplies generate higher-than-specificationHvoltages during magnet ramping and quenches that can damage the GPIB-USB
interface.LThe slow controls software includes a custom kernel driver for the PCI 6025EMcard, a custom-developed server written in C++ that listens for commands on aHUNIX FIFO and a client perl script that provides a user interface to theNserver. The client-server interface ensures that multiple users can access andMcontrol the slow control system simultaneously, but also ensures that controlIis not transferred during critical segments of a run. It also facilitates?remote login for monitoring and controlling of the experiment. %\subsubsection{Trigger logic}R%%The following was Chris trying to make sense of this paragraph before canning itM%To facilitate data storage the trigger system is set up to direct the DAQ toM%digitize only signals with a voltage pulse greater than a threshold voltage.P%The analog signal pulse is split by a LeCroy 428F linear fan-in/fan-out module.N%Two of the analog pulses are sent {\bf through a delay of \fixme XXX~ns or isN%this internal to the digitizer?} and delivered directly to the digitizer cardP%{\bf \fixme Is this for a high gain/low gain channel? is one of these amplifiedJ%then?} Another linear copy of the analog pulse is delivered to a PhillipsJ%Scientific model 708 discriminator that generates the initial logic pulseP%indicating the start of the signal. This logic pulse is sent through a PhillipsM%Scientific model 794 Gate and Delay generator to produce a 100~ns logic gateL%that serves at the trigger to signal the digitizer to begin a digitization.O%{\bf \fixme What is the LeCroy 662 and what is it doing to the trigger logic?}J%A secondary 4.66 $\mu$s inverted gate is generated along with the primaryN%trigger gate to inhibit the system from responding to any further triggers of.%the system during an ongoing  digitization.  %%%The Original paragraph%The relatively simple trigger logic functions as follows.  Signals, with a voltage range up to $\sim3$V(?) from each main-detector \PMT\ are linearly split (LeCroy 428F).  Part of the signal goes directly (no delay) to two channels of the GaGe cards.  The other part goes to a discriminator set at approximately xx mV (Phillips Scientific 708).  The discriminator signals separately generate  $\sim$100 ns gates (PS 794), which though an AND gate form the primary trigger (LeCroy 662).  A separate $4.66 \mu$s inverted gate is generated from the primary trigger and used as a veto (ANDed with the trigger) to allow the GaGe card sufficient time to digitize an event as well as reduce the rate of digitization of large events with many trailing photons.  The primary trigger is again ANDed with the DAQ control logic described in Section xxx before triggering the GaGe Cards.  Once triggered, all four GaGe channels are digitized.  Coincidences between the main detector and either the reference pulse or the muon veto are determined in software."\subsubsection{Trace digitization}MSignals from four channels are digitized by two hardware synchronized CS12502GGage cards installed in two PCI slots of the \DAQ\ computer.  Each cardOprovides two channels of 12 bit data sampled at the rate of 500 megasamples perIsecond. {\bf \fixme what is the dynamic range?}  Hardware synchronizationEinsures that all four channels are synchronized and that they respondLsimultaneously to a common trigger.  The first two channels are used for theOsplit channel output of the main detector, the third is contains the linear sumIof the muon veto paddles, and the forth contains a reference pulser.  TheKacquisition of waveforms from the four channels is triggered by the fallingKedge of an externally generated trigger signal.  Each of the four waveformsMconsists of 704 12-bit values (total time $\approx 1.4\,\mu\textrm{sec}$), ofOwhich 192 values are pre-trigger and 512 events are post-trigger.  A time stampGbased upon a 137.5\,Mhz on-board oscillator ($\approx 7\,\textrm{nsec}$Oresolution) is also recorded with each acquisition of four waveforms.  The timeJstamp and waveforms are stored in binary form in files along with a headerJcontaining all of the configuration parameters for the Gage cards in asciiLform.  The PC-based DAQ interfaces with the Mac-based slow controls via fourWdigital lines of which two are output (\DAQ$\rightarrow$slow control) and two are input (slow control$\rightarrow$\DAQ).LThe following sequence of steps describe what happens on the \DAQ\ after theMacquire button is pushed: 1) Header information is written to a new datafile.O2) The status of the magnet (on/off) is read on one of the two input lines fromJthe slow control machine and stored in the header information. 3) Once theNdigitizing cards are successfully armed, a signal is raised to inform the slowLcontrol that the cards are armed.  4) A wait begins for a raised signal fromIthe slow control indicating that data taking can proceed.  5) ImmediatelyKbefore the digitizing cards are enabled, a signal is raised indicating thatNdata taking is occurring.  6) As soon as the on-board memory of the digitizingMcards is filled (typically after 10000 digitizations), acquisition is stoppedHand a signal indicating that data taking has stopped is sent to the slowIcontrol.  7) The data are transfered from the card to computer memory andOwritten to disk (optionally, histograms are generated and plotted).  8) At thisOpoint, one of the input lines is checked to see whether or not data acquisitionNshould be stopped.  If not, the cycle begins anew from step 5.  Data taking isNstopped by pushing the stop button on the panel, by the slow control, or afterKa fixed period of time.  In the last two cases, the program closes the data+file and the cycle begins anew from step 1.\section{Performance}NThe goal of the experiment is to provide a measurement of the neutron lifetimeEwith as large of a signal-to-background ratio as possible with as fewIsystematics as possible. The realization of this goal depends on many keyMperformance parameters all being optimized in unison. Operationally, the heatLload to 4.2~K and to the dilution refrigerator and experimental cell must beMminimized to maintain stable operating temperatures and minimize upscatter ofLUCN from the trap. It should also be small enough that the supply demands ofLliquid helium can be met. Further the operation of the trap that necessarilyNproduces large magnetic fields also impacts other local experiments minimally.LTo directly meet the goal of high signal-to background, the \UCN\ productionLrate must be sufficiently high and the background rate from external sourcesFmust be relatively low and/or shielded from. An important factor is anIassesment of the detection efficiency for neutron decays. To be sure thatNsignal is truely neutron decay, one must also evaluate all possible sources ofsystematic uncertainty.  FThe following is just to guide my thoughts, I'll remove it soon (CMO):,Key performance parameters of the apparatus:Operational stability/cost\begin{itemize}\item heat loads to 4.2K#\item electrical power consumption?+\item Magnetic field shielding in guidehall
\end{itemize}$Sensitivity -- Signal to Background:\begin{itemize}\item \UCN production rate G\item Background rate (sources of backgrounds and shielding efficiency)!\item Light collection efficiency
\end{itemize}Sources of Systematics:\begin{itemize}.\item Helium purity (brief for separate paper),\item Heat load to the cell - \UCN upscatter\item Marginally trapped \UCN\item Magnetic field stability%\item Magnetic field shielding (PMTs)
\end{itemize}#\subsection{Heat Load Calculations}\label{sec:HeatLoad}LThe cell heat load comes from blackbody radiation, conduction heat from cellNsupports and sensor wires, neutron beam heating and eddy current heating.  TheOestimated heat load from each source is summarized in Table~\ref{tab:cellheat}.OThe eddy current heating has the largest peak heat input.  Fortunately, it onlyKhappens during magnet ramps that take place about 8~\% of the total runningMtime.  The average heat load will be 250~$\mu$W, with a peak heatload of 2~mWduring magnet ramps.        \begin{table}[h]	\begin{center}A	\caption{Summary of estimated cell heat loads.  The eddy currentB	calculation assumes a magnet ramp time of 200~s.  Duration is the=	percentage of time during a run when the heat source is on.}	\vskip 1em	\begin{tabular}{|c|c|c|}	    \hline;	     Heat Source  & Heat input ($\mu$W) & Duration (\%) \\	    \hline(	     Blackbody radiation & 48 &  100 \\1	     Cell supports and sensor wires & 5 & 100 \\	     Neutron beam & 44 & 40 \\4	     Eddy current in cell wall (CuNi) & 1000 & 8 \\7	     Eddy current in buffer cell (copper) & 1200 & 8\\	    \hline	\end{tabular}	\label{tab:cellheat}
	\end{center}    \end{table}KThe heat load onto the 4~K can comes from backbody radiation, the cryogenicMsupport posts, HTS current leads, eddy current heating in the solenoid forms,Loperation of the dilution refrigerator, and heat input down the two verticalNtowers.  Table~\ref{tab:4KHeatLoad} summarizes the estimates for each of theseMheat sources.  The blackbody radiation is estimated assuming an emissivity ofN0.1 for surfaces covered by super-insulation and 1 for the neutron and opticalOwindows.  The eddy current heating in the aluminum form is estimated assuming aOmagnet ramp time of 200~s.  The liquid helium consumption from operation of the=dilution refrigerator was measured during a stand alone test.     \begin{table}[t]	\begin{center}E	\caption{Summary of estimated cell heat loads to 4.2K surfaces.  TheH	eddy current calculation assumes a magnet ramp time of 200~s.  DurationD	is the percentage of time during a run when the heat source is on.}	\vskip 1em"	\renewcommand{\arraystretch}{1.5}	\begin{tabular}{|c|c|c|c|}	\hlineB	 Heat Source  & Heat input (W) & Duration & Equiv. LHe (l/day) \\	\hline+	 Blackbody Radiation & 0.8 & 100\% & 27 \\'	 Cryogenic Posts & 0.8 & 100\% & 27 \\)	 Fermilab HTS leads & 1.4 & 100\% & 46\\+	 Low current HTS leads & 0.1 & 100\% & 3\\,	 Eddy current in solenoids & 5 & 8\% & 13\\&	 Dilution fridge& 0.6 & 100\% & 20 \\'	 Vertical towers & 0.6 & 100\% & 20 \\	\hline	\end{tabular}	\label{tab:4KHeatLoad}
	\end{center}    \end{table}IThe total heat input to the 4.2~K shields based on the above estimates isM4.7~W, corresponding to a liquid helium consumption of 156~l/ day.  To reduceOthis consumption rate, two cryocoolers each providing 1.5~W of cooling power atM4.2~K are used in the apparatus and are shown in Figure~\ref{fig:dewar}.  OneIconnects directly to the HTS current leads, and the other attaches to theOcryogenic posts.  If the cooling power of the cryocooler can be fully utilized,Lwe estimate the liquid helium boil-off can be reduced to roughly 60~l/day.      LIn practice, the helium consumption is between 80~l/day and 100~l/day duringLday-to-day operations depending on the running configuration of the magnets.MTo additionally reduce the helium consumption, a commercial reliquifier plantLhas been added to the system and provides 30~l/day of liquid by recondensing#the helium boiloff from the system.(\subsection{UCN Production {\bf Pieter}}*\subsection{Light Collection {\bf Chris} },Is there supporting data from the data runs?:\subsection{Shielding efficiency/Backgrounds {\bf Pieter}}H\subsection{Helium Purity {\bf Mike} {\it As it related to data taking}}%\subsection{Above Threshold Neutrons}OA neutron with enough kinetic energy can overcome the magnetic potential of theMtrap and potentially interact with the walls of the cell.  These neutrons areMtermed \MTN. When a neutron interacts with the wall there is some probabilityHthat the neutron will be lost.  Therefore the \MTN s are subjected to anOadditional loss mechanism resulting in a systematic effect when determining theGneutron lifetime.  A method of decreasing the effect of \MTN s has beenFdeveloped.  By briefly ramping down the magnetic field strength of theKquadrupole magnet the orbit size for the neutrons in the trap is increased.MWhile the field is ramped down the neutrons interact more frequently with theKcell walls.  Some fraction of the neutrons that were fully trapped are thenKable to interact with the wall when the field strength is low, resulting inOadditional lost neutrons and therefore a decreased statistical sensitivity. TheN\MTN s are preferentially purged during the low field state. Varying the fieldLstrength and the duration that the field is held in the minimum state allowsMfor a reduction in the \MTN\ population to any desired level.  The goal is toMtake data in a running configuration where little to no systematic correctionJneeds to be applied to the data.  A monte-carlo simulation is performed toJestimate the systematic effect due to \MTN s in the apparatus in any givenLramping configuration. In this manner the ramping configuration is chosen toOminimize the systematic correction. A detailed benchmarking and verification of>the Monte Carlo is underway, and will be published seperately.NThe Monte Carlo uses a detailed magnetic field map and a symplectic integratorLto integrate the equations of motions under the adiabatic condition.  When aMneutron crosses a boundary of the cell wall an interaction is simulated.  TheKwall interaction consists of two parts.  First we calculate the probabilityJthat the neutron was reflected; Second we calculate the direction that theNneutron leaves the wall interaction.  The reflection probability is calculatedOusing a multi-layered wall potential and the optical model as described in [1].LWhere the real and imaginary optical potentials and material thicknesses areOused to calculate the reflection probability as a function of energy by solvingMa simple one dimensional scattering problem from a series of step potentials.FThe exit angle of the neutron is sampled using the lambertian model, aGperfectly diffuse scattering model.  The use of a diffuse wall model isJmotivated by the highly irregular surface structure of the cell materials.OUsing this information realistic trajectories of \UCN\ inside the apparatus areNsimulated.  Each neutron is tracked until the survival probability falls belowM1e-15 at which point the neutron is discarded and the next simulation begins.OUltimately the Monte Carlo outputs a survival probability as a function of timeNfor an ensemble of neutrons that solely interact with the walls (no beta decayEor other loss mechanisms).  Using this survival probability curve theLsystematic correction due to \MTN s can be estimated.  Simulations have beenMpreformed with a preliminary version of the Monte Carlo predicting systematicMcorrections ranging from a few seconds to as large as 72 seconds depending onLthe running configuration.  The \MTN s do systematically reduce the neutronsOtrap lifetime, however with the assistance of a detailed Monte Carlo simulationLit has been shown that the size of the systematic effect can be limited to aFnegligible level by carefully choosing the field ramping parameters.  5��